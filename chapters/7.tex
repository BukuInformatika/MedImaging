\section{Nomor 1}
\subsection{Bagian a}
Decay constant didalam $\lambda$ ditemukanlah:
\begin{displaymath} 
\lambda = \frac {0.693}{T _{1/2}} \approx 1.4808 \times 10 ^{-5} sec ^{-1}
\end{displaymath}
Radioactivity A nya yaitu:
\begin{displaymath}
A = \lambda \textit{N} = 1.4808 \times 10 ^(-5) \times 10^9 = 1.4808 \times 10^4 dps
\end{displaymath}
\subsection{Bagian b}
Sejak $\textit{N}^t = \textit{N}_0e^{- \lambda t}$, maka:
\begin{displaymath}
\textit{N}_{24}h = 10^9 \times exp(-1.4808 \times 10^{-5} \times 24 \times 3600) \approx 2.78 \times 10^8 atoms.
\end{displaymath}

\section{Nomor 2}
$A_0 = 1 Ci = 3.7 \times 10^{10}Bq$ dan $A_\textit{t} = A_0e^{\lambda \textit{t}} = 1Bq$. Jadi:
\begin{displaymath}
e^{- \lambda \textit{t}} = \frac {1}{3.7 \times 10^{10}} = 2.7 \times 10^{-11}
\end{displaymath}
dimana disimpulkan:
\begin{displaymath}
- \lambda \textit{t} = In (2.7 \times 10^{-11} = -24.334
\end{displaymath}
\begin{displaymath}
\to t = \frac {24.334}{\lambda}
\end{displaymath}
Ketika $\textit{T}_{1/2} = \frac {0.693}{\lambda} = \tau$, kita mempunyai $\lambda = \frac {0.693}{\tau}$, dan $\textit{t} = 35.114 \tau$. Dibutuhkan $\textit{t} = 35.114 \tau$ untuk sampel radioaktif dengan aktivitas sebenyak 1 Ci untuk kerusakan untuk aktivitas sebanyak 1 Bq jika sebagian kehidupannya yaitu $\tau$

\section{Nomor 3}
DF diartikan sebagai DF = $e^{- \lambda \textit{t}}$. Dan kerusakan konstan $\lambda$ diberikan oleh:
\begin{displaymath}
\frac {A_{1/2}}{A_0} = \frac {1}{2} = e^{- \lambda \textit{T} _{1/2}}
\end{displaymath}
Mengambil logaritma natural dari equation diatas maka $- \lambda \textit{T}_{1/2}= -ln 2 = -0.693$, dan $\lambda = \frac {0.693}{\textit{T}^{1/2}}$