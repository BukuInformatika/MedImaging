\section{Jawaban Nomor 1}
Karena detektor dirancang untuk menghentikan 75\% foton, kami memiliki \begin{displaymath} 0,75 = e-\mu d\end{displaymath} dimana d adalah detektor ketebalan. Karena itu kita punya
\center
\begin{displaymath}
for NaI(Tl) : d = ln0.75/(-\mu) 
\end{displaymath}
\begin{displaymath}
= (ln0.75)/(-0.343) 
\end{displaymath}
\begin{displaymath}
= 0.8387 cm, 
\end{displaymath}
\begin{displaymath}
for BGO : d = ln0.75/(-\mu) 
\end{displaymath}
\begin{displaymath}
= (ln0.75)/(-0.964)
\end{displaymath} 
\begin{displaymath}
= 0.2984 cm.
\end{displaymath}
\section{Jawaban Nomor 2}
\begin{enumerate}
	\item Tidak ada kolimator. Dalam pemindai PET, seseorang harus dapat mendeteksi kebetulan di berbagai sudut.
	\item Anda harus menambahkan detektor kebetulan.
	\item Dari geometri \begin{displaymath}
				\alpha = tan^{-1}\frac {0.15 m}{0.75 m}
					= 11.31^\circ.
				\end{displaymath}
\end{enumerate}
\section{Jawaban Nomor 3}
\begin{enumerate}
\item Keliling lingkaran adalah $\pi D = 1.5\pi \approx 4.712 m.$
\subitem Lebar detektor perkiraan adalah $\frac{4.712 m}{1,000} = 4.712 mm.$
Detektor dangkal kurang efisien untuk menghentikan foton gamma, tetapi foton gamma yang masuk dari segala arah dapat dideteksi secara sama. Detektor dalam lebih efisien, tetapi lebih selektif arah.
\item Deteksi kebetulan dalam PET digunakan untuk menentukan arah perjalanan dua foton gamma back-to-back, dan karenanya untuk memutuskan jalur mana radioaktivitas terjadi. Kebetulan diasumsikan jika dua peristiwa terjadi dalam 2-12 ns dalam pemindai PET yang khas. Karena radioaktivitas tidak selalu terjadi di pusat pemindai PET, waktu perjalanan dua foton gamma back-to-back tidak sama.  Jika interval waktu terlalu kecil, radioaktif tidak akan terdeteksi. Jika baterai permanen terpasang, hamburphoton akan tetap dihitung. Juga, peluruhan twoormoredistinctpositron mungkin dicampur bersama, dan garis kebetulan tidak dapat lagi ditentukan dengan benar.
\end{enumerate}