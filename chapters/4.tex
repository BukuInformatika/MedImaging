\section{Jawaban No. 1}
Radiasi karakteristik dihasilkan oleh elektron yang turun ke tingkat energi yang lebih rendah (lebih banyak orbit bagian dalam) setelah mereka diekskresikan menjadi lebih tinggi terhadap energiterat (lebih banyak oksigen). Pembeda inferensiil harus dilakukan dari x-ray radiasi karakteristik. Karena elektron-elektron yang ada secara eksplisit memiliki energi yang berbeda untuk atom yang diagregasi, radiasi karakteristik hanya dapat dipancarkan pada kumpulan tingkat energi diskrit dalam spektrum EM. Oleh karena itu, spektrum intensitas untuk radiasi karakteristik terdiri dari spektrum diskrit yaitu, garis spektral. Di sisi lain, Brstrstrahlungradiation diputuskan oleh interaksi darimemilih elektron dengan atom inti. Khususnya, nukleus, yang memiliki muatan positif, akan cenderung menarik elektron, memiliki muatan negatif, menyebabkan elektron melambat dan terdefleksi dari jalur aslinya. Elektron kehilangan energi sebagai akibatnya, yang diradiasikan dengan sinar-x dengan energi yang sama dengan yang dikeluarkan oleh elektron. Energi listrik dapat kehilangan energi secara keseluruhan, dengan menabrak inti atom, atau jumlah yang lebih kecil, dengan defleksi yang lebih kecil. Oleh karena itu, tidak seperti radiasi karakteristik, radiasi energi tidak terputus. Karena energi yang lebih rendah, dan tumbukan langsung dengan nukleus sangat tidak mungkin, spektrum bremsstrahlung nol pada energi kejadian elektron dan tumbuh lebih besar dengan penurunan energi.

\section{Jawaban No. 2}
\begin{enumerate}
\item Panjang gelombang frekuensi pada gelombang EM akan berelasi dengan rumus $\lambda =  \frac{c}{v}$
\subitem Dimana \begin{displaymath} \mathit{c} = 3.0 \times 10^{8} meters/sec \end{displaymath} pada kecepatan cahaya. Untuk $\lambda$ = 4 nanometers, kita akan mendapatkan :
\subitem  $$ v =\frac{c}{\lambda}$$
\subitem  $$ = \frac{ 3.0 \times 10^{8}m/s}{4 \times 10^{-9}m}$$
\subitem \begin{displaymath}= 7.5 \times 10^{16}Hz\end{displaymath}
\subitem$$\lambda = 400{nanometers}$$ 
\subitem \begin{displaymath}v = 7.5 \times 10^{14}Hz\end{displaymath}
\subitem Jadi panjang jarak frekuensi sinar ultraviolet adalah
\begin{displaymath}
7.5 \times 10^{14}Hz - 7.5 \times 10^{16}Hz.
\end{displaymath}
\item Energi photon dapat diasumsikan dengan rumus $E = hv$ dimana nilai $h = 6.626 \times 10^{-34} Joule-see$ Untuk sinar ultraviolet dengan frekuensi $v = 7.5 \times 10^{14}Hz$
\subitem$E = hv$
\subitem $=  6.626 \times 10^{-34} \times 7.5 \times 10^{14}$
\subitem $=  4.97 \times 10^{-19} Joule$
\subitem $1 eV = 1.6 \times 10^{-19} Joule$
\subitem $E = 4.97 \times 10^{-19} Joule = 3.1 eV$
\subitem Persamaan untuk sinar ultraviolet dengan frekuensi $v = 7.5 \times 10^{16}Hz$ dengan energi $E = 310 eV$. Jadi photon energi dengan rentan jarak untuk sinar ultraviolet dengan 3.1-310 eV 
\end{enumerate}

\section{Jawaban No. 3}
Radiasi dengan energi yang lebih baik atau setara dengan 13.6 eV bisa dianggap dengan ionizing radiation. Hal yang mudah untuk di kalkulasikan ketika frekuensi sinar ultraviolet adalah $v = 3.284 \times 10^{15} Hz$, proton enerigi dapat didefinisikan dengan $E = hv = 13.6 eV$ . Jadi sinar ultraviolet akan mengionisasi radiasi ketika frekuensi lebih baik atau setara dengan nilai $v_{0} = 3.284 \times 10^{15}Hz$. Sinar ultraviolet dengan frekuensi yang rendah tidak akan mengionisasikan radiasi. Atau nilai ekuivalen, ketika jarak gelombang lebih besar daripada $\lambda_{0} = \frac{e}{v_{0}} = 91.35 nanometers$, sinar ultraviolet tidak terionisasi radiasi. 