\section{Nomor 1}
Ketika kita memilih radionuklida didalam obat nuklir, masalah berikut harus dipertimbangkan:
\begin{itemize}
  \item Radionuklida haruslah "bersih" dari penghasil sinar gamma, yang berarti mereka tidak memancarkan partikel alfa atau beta.
  \item Radionuklida harus memancarkan sinar gamma dengan energi yang sesuai. Energi tidak boleh terlalu rendah karena sinar gamma energi rendah lebih mungkin diserap oleh tubuh; Oleh karena itu, tingkatkan dosis pasien tanpa berkontribusi pada efek negatifnya. Selain itu, energinya tidak boleh terlalu tinggi karena sinar gamma berenergi tinggi cenderung tidak terdeteksi.
  \item Radionuklida harus memiliki waktu paruh dalam urutan menit hingga jam.
  \item Radionuklida harus bermanfaat dan aman untuk dilacak di dalam tubuh.
  \item Radionuklida harus memancarkan sinar gamma se-monokromatik mungkin.
\end{itemize}

\section{Nomor 2}
\subsection{Bagian a}
Catatan bahwa 20\% jendela nadi tingkat tinggi adalah 10\% di sisi lain.
\begin{displaymath}
150 KeV \times 0.1 = 15 KeV ,
\end{displaymath}
\begin{displaymath}
150 KeV - 15 KeV = 135 KeV.
\end{displaymath}
Semenjak
\begin{displaymath}
\textit{hv'} = \frac {\textit{hv}}{1 + \frac {\textit{hv}}{mo\textit{c}^2}(1 - cos \theta)}
\end{displaymath}
kita mempunyai
\begin{displaymath}
135 KeV = \frac {140 KeV}{1 + \frac {140 KeV}{511 KeV}(1 - cos \theta)}
\end{displaymath}
untuk penyelesaian $\theta$, kita mendapatkan $\theta = 30.14 ^\circ$.
\subsection{Bagian b}
Untuk Untuk jendela yang berpusat di photopeak, sudut maksimum yang dapat diterima untuk foton 140 keV adalah 53,54$^\circ$. Lakukan perhitungan yang sama, kita dapat melihat bahwa foton dengan energi hv = 364 keV dapat tersebar dengan sudut $\theta$ = 32,43$^\circ$ dan masih dapat diterima oleh jendela 20\% yang berpusat di photopeak.

\section{Nomor 3}
Dari perhitungan yang sederhana, kita mempunyai output dari PMT nya yaitu:
\begin{displaymath}
\alpha _1 = 21.10 \alpha _2 = 21.10 \alpha _3 = 12.13
\end{displaymath}
\begin{displaymath}
\alpha _4 = 21.10 \alpha _5 = 21.10 \alpha _6 = 12.13
\end{displaymath}
\begin{displaymath}
\alpha _7 = 12.13 \alpha _8 = 12.13 \alpha _9 = 8.13
\end{displaymath}