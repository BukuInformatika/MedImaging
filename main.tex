\documentclass{wileySix}

\usepackage{graphicx}
\usepackage{listings}

\usepackage{color}

\definecolor{codegreen}{rgb}{0,0.6,0}
\definecolor{codegray}{rgb}{0.5,0.5,0.5}
\definecolor{codepurple}{rgb}{0.58,0,0.82}
\definecolor{backcolour}{rgb}{0.95,0.95,0.92}

\lstdefinestyle{mystyle}{
    backgroundcolor=\color{backcolour},
    commentstyle=\color{codegreen},
    keywordstyle=\color{magenta},
    numberstyle=\tiny\color{codegray},
    stringstyle=\color{codepurple},
    basicstyle=\footnotesize,
    breakatwhitespace=false,
    breaklines=true,
    captionpos=b,
    keepspaces=true,
    numbers=left,
    numbersep=5pt,
    showspaces=false,
    showstringspaces=false,
    showtabs=false,
    tabsize=2,
    language=TeX,
	caption=Coy kasih caption dong,
	label={lstlisting_nya_ada_yang_belum_dikasih_label}
}

\lstset{style=mystyle}

\usepackage{w-bookps}

\setcounter{secnumdepth}{3}

\setcounter{tocdepth}{2}


\docropmarks


\newcommand{\VT}[1]{\ensuremath{{V_{T#1}}}}


\newbox\sectsavebox
\setbox\sectsavebox=\hbox{\boldmath\VT{xyz}}



\begin{document}


\booktitle{Cerdas Menguasai Latex}
\subtitle{Dalam 24 Jam}

\authors{Rolly M. Awangga\\
\affil{Informatics Research Center}}

\offprintinfo{Cerdas Menguasai Latex, First Edition}{Rolly M. Awangga}


\halftitlepage

\titlepage


\begin{copyrightpage}{2019}
%Survey Methodology / Robert M. Groves . . . [et al.].
%\       p. cm.---(Wiley series in survey methodology)
%\    ``Wiley-Interscience."
%\    Includes bibliographical references and index.
%\    ISBN 0-471-48348-6 (pbk.)
%\    1. Surveys---Methodology.  2. Social 
%\  sciences---Research---Statistical methods.  I. Groves, Robert M.  II. %
%Series.\\
%
%HA31.2.S873 2007
%001.4'33---dc22                                             2004044064
\end{copyrightpage}

\dedication{`Jika Kamu tidak dapat menahan lelahnya belajar,
Maka kamu harus sanggup menahan perihnya Kebodohan.'
~Imam Syafi'i~}

\begin{contributors}
\name{Rolly Maulana Awangga,} Informatics Research Center., Politeknik Pos Indonesia, Bandung,
Indonesia



\end{contributors}

\contentsinbrief
\tableofcontents
\listoffigures
\listoftables
\lstlistoflistings


\begin{foreword}
Sepatah kata dari Kaprodi, Kabag Kemahasiswaan dan Mahasiswa
\end{foreword}

\begin{preface}
Buku ini diciptakan bagi yang awam dengan git sekalipun.

\prefaceauthor{R. M. Awangga}
\where{Bandung, Jawa Barat\\
Februari, 2019}
\end{preface}


\begin{acknowledgments}
Terima kasih atas semua masukan dari para mahasiswa agar bisa membuat buku ini 
lebih baik dan lebih mudah dimengerti.

Terima kasih ini juga ditujukan khusus untuk team IRC yang 
telah fokus untuk belajar dan memahami bagaimana buku ini mendampingi proses 
Intership.
\authorinitials{R. M. A.}
\end{acknowledgments}

\begin{acronyms}
\acro{ACGIH}{American Conference of Governmental Industrial Hygienists}
\acro{AEC}{Atomic Energy Commission}
\acro{OSHA}{Occupational Health and Safety Commission}
\acro{SAMA}{Scientific Apparatus Makers Association}
\acro{EPS}{Encapsulated PostScript}
\acro{HTBP}{Here Tab Bottom Paragraph}
\acro{IDE}{Integrated Development Environment}
\acro{GPL}{General Public Lisence}
\end{acronyms}

\begin{glossary}
\term{git}Merupakan manajemen sumber kode yang dibuat oleh linus torvald.

\term{bash}Merupakan bahasa sistem operasi berbasiskan *NIX.

\term{linux}Sistem operasi berbasis sumber kode terbuka yang dibuat oleh Linus Torvald

\term {compile}Analisis pada kode program untuk mengubah komputer bentuk langsung eksekusi dari program

\term {script}Bahasa yang digunakan untuk menerjemahkan setiap perintah dalam situs pada saat diakses.

\term {listing}Teks yang berisi daftar item berupa perintah langkah-langkah membuat program

\term {bullets} Satu tanda yang dipakai untuk memberikan gambar atau lambang pada latex
\end{glossary}

\begin{symbols}
\term{A}Amplitude

\term{\hbox{\&}}Propositional logic symbol 

\term{a}Filter Coefficient

\bigskip

\term{\mathcal{B}}Number of Beats
\end{symbols}

\begin{introduction}

%% optional, but if you want to list author:

\introauthor{Rolly Maulana Awangga, S.T., M.T.}
{Informatics Research Center\\
Bandung, Jawa Barat, Indonesia}

Pada era disruptif  \index{disruptif}\index{disruptif!modern} 
saat ini. git merupakan sebuah kebutuhan dalam sebuah organisasi pengembangan perangkat lunak.
Buku ini diharapkan bisa menjadi penghantar para programmer, analis, IT Operation dan Project Manajer.
Dalam melakukan implementasi git pada diri dan organisasinya.

Rumusnya cuman sebagai contoh aja biar keren\cite{awangga2018sampeu}.

\begin{equation}
ABC {\cal DEF} \alpha\beta\Gamma\Delta\sum^{abc}_{def}
\end{equation}

\end{introduction}

%%%%%%%%%%%%%%%%%%Isi Buku_

\chapter{4}



\chapter{5}



\chapter{6}



\chapter{7}



\chapter{8}



\chapter{9}



\chapter{10}



\chapter{11}
\section{Jawaban No.1}


\section{Jawaban No.2}
Untuk M-mode dapat mengambil sampel sinyal hingga 3,700/2 = 1,850. sedangkan untuk B-Mode tidak dapat mengambil frekuensi lebih tinggi dari 14,4/2=7,2 second tanpa memasukan aliasing

\section{Jawab No.3}
Sebagai menghantarkan gelombang suara yang direflesikan antara jaringan yang berbeda atau tersebar dari struktur yang lebih kecil. secara khusus suara akan dipantulkan dimana saja ada perubahan impedansi ditubuh.Pada transduser tersebut akan memungkinkan suara untuk ditransmisikan secara efisien ke dalam tubuh (seringkali lapisan karet, suatu bentuk pencocokan impedansi).


\bibliographystyle{IEEEtran}
\bibliography{references}


\printindex
\end{document}

