\documentclass{wileySix}

\usepackage{graphicx}
\usepackage{listings}

\usepackage{color}
 
\definecolor{codegreen}{rgb}{0,0.6,0}
\definecolor{codegray}{rgb}{0.5,0.5,0.5}
\definecolor{codepurple}{rgb}{0.58,0,0.82}
\definecolor{backcolour}{rgb}{0.95,0.95,0.92}
 
\lstdefinestyle{mystyle}{
    backgroundcolor=\color{backcolour},   
    commentstyle=\color{codegreen},
    keywordstyle=\color{magenta},
    numberstyle=\tiny\color{codegray},
    stringstyle=\color{codepurple},
    basicstyle=\footnotesize,
    breakatwhitespace=false,         
    breaklines=true,                 
    captionpos=b,                    
    keepspaces=true,                 
    numbers=left,                    
    numbersep=5pt,                  
    showspaces=false,                
    showstringspaces=false,
    showtabs=false,                  
    tabsize=2,
    language=TeX,
	caption=Coy kasih caption dong,
	label={lstlisting_nya_ada_yang_belum_dikasih_label}
}
 
\lstset{style=mystyle}

\usepackage{w-bookps}

\setcounter{secnumdepth}{3}

\setcounter{tocdepth}{2}


\docropmarks


\newcommand{\VT}[1]{\ensuremath{{V_{T#1}}}}


\newbox\sectsavebox
\setbox\sectsavebox=\hbox{\boldmath\VT{xyz}}



\begin{document}


\booktitle{Cerdas Menguasai Latex}
\subtitle{Dalam 24 Jam}

\authors{Rolly M. Awangga\\
\affil{Informatics Research Center}}

\offprintinfo{Cerdas Menguasai Latex, First Edition}{Rolly M. Awangga}


\halftitlepage

\titlepage


\begin{copyrightpage}{2019}
\input{info/copyrightpage}
\end{copyrightpage}

\dedication{`Jika Kamu tidak dapat menahan lelahnya belajar, 
Maka kamu harus sanggup menahan perihnya Kebodohan.'
~Imam Syafi'i~}

\begin{contributors}
\input{info/contributors}
\end{contributors}

\contentsinbrief
\tableofcontents
\listoffigures
\listoftables
\lstlistoflistings


\begin{foreword}
\input{info/foreword}
\end{foreword}

\begin{preface}
\input{info/preface}
\end{preface}


\begin{acknowledgments}
\input{info/acknowledgments}
\end{acknowledgments}

\begin{acronyms}
\input{info/acronyms}
\end{acronyms}

\begin{glossary}
\term{git}Merupakan manajemen sumber kode yang dibuat oleh linus torvald.

\term{bash}Merupakan bahasa sistem operasi berbasiskan *NIX.

\term{linux}Sistem operasi berbasis sumber kode terbuka yang dibuat oleh Linus Torvald

\term {compile}Analisis pada kode program untuk mengubah komputer bentuk langsung eksekusi dari program

\term {script}Bahasa yang digunakan untuk menerjemahkan setiap perintah dalam situs pada saat diakses.

\term {listing}Teks yang berisi daftar item berupa perintah langkah-langkah membuat program

\term {bullets} Satu tanda yang dipakai untuk memberikan gambar atau lambang pada latex
\end{glossary}

\begin{symbols}
\input{info/symbols}
\end{symbols}

\begin{introduction}
\input{info/introduction}
\end{introduction}

%%%%%%%%%%%%%%%%%%Isi Buku_

\chapter{Editor dan Compiler}
\input{chapters/0}

\chapter{Pengaturan Paragraf}
\input{chapters/1}

\chapter{Menambahkan Gambar}
\input{chapters/2}



\bibliographystyle{IEEEtran} 
\bibliography{references}


\printindex
\end{document}

